\documentclass[12pt, letterpaper]{article}

\title{Implementations of 30+ Cryptography Algorithms in Rust}
\author{Saurabh Gupta}
\date{June 2024}

\begin{document}

\maketitle

\begin{abstract}

This is the abstract of your document.
It should provide a brief summary of your work,
typically in about 150-300 words.
The abstract gives readers a quick overview of 
your paper's main points and conclusions.

\end{abstract}

\tableofcontents
\pagebreak

\section{Introduction}

Cryptography is the practice of secure communication 

\pagebreak

\section{Simple Substitution}
Simple substitution ciphers are basic encryption techniques where each letter in the plaintext is replaced by another letter or symbol in the ciphertext.

\subsection{Additive}
The additive cipher, also known as Caesar cipher, shifts each letter of the alphabet by a fixed number of positions.

\begin{equation}
E(x) = (x + k) \bmod 26
\end{equation}

Where $E(x)$ is the encrypted letter, $x$ is the original letter (as a number from 0-25), and $k$ is the shift key.

\subsection{Multiplicative}
In a multiplicative cipher, each letter is multiplied by a fixed number (modulo 26).

\begin{equation}
E(x) = (ax) \bmod 26
\end{equation}

Where $a$ is the multiplication key, which must be coprime to 26.

\subsection{Affine}
The affine cipher combines both additive and multiplicative methods:

\begin{equation}
E(x) = (ax + b) \bmod 26
\end{equation}

Where $a$ and $b$ are the keys, and $a$ must be coprime to 26.

% \subsection{Hill}
% The Hill cipher uses matrix multiplication to encrypt blocks of letters. For a 2x2 matrix:

% \begin{equation}
% \begin{pmatrix} 
% c_1 \\ c_2 
% \end{pmatrix} = 
% \begin{pmatrix} 
% k_{11} & k_{12} \\ 
% k_{21} & k_{22}
% \end{pmatrix}
% \begin{pmatrix} 
% p_1 \\ p_2 
% \end{pmatrix} \bmod 26
% \end{equation}

% Where $c_i$ are ciphertext letters, $p_i$ are plaintext letters, and $k_{ij}$ form the key matrix.


\end{document}

